\PassOptionsToPackage{unicode=true}{hyperref} % options for packages loaded elsewhere
\PassOptionsToPackage{hyphens}{url}
%
\documentclass[
]{article}
\usepackage{lmodern}
\usepackage{amssymb,amsmath}
\usepackage{ifxetex,ifluatex}
\ifnum 0\ifxetex 1\fi\ifluatex 1\fi=0 % if pdftex
  \usepackage[T1]{fontenc}
  \usepackage[utf8]{inputenc}
  \usepackage{textcomp} % provides euro and other symbols
\else % if luatex or xelatex
  \usepackage{unicode-math}
  \defaultfontfeatures{Scale=MatchLowercase}
  \defaultfontfeatures[\rmfamily]{Ligatures=TeX,Scale=1}
\fi
% use upquote if available, for straight quotes in verbatim environments
\IfFileExists{upquote.sty}{\usepackage{upquote}}{}
\IfFileExists{microtype.sty}{% use microtype if available
  \usepackage[]{microtype}
  \UseMicrotypeSet[protrusion]{basicmath} % disable protrusion for tt fonts
}{}
\makeatletter
\@ifundefined{KOMAClassName}{% if non-KOMA class
  \IfFileExists{parskip.sty}{%
    \usepackage{parskip}
  }{% else
    \setlength{\parindent}{0pt}
    \setlength{\parskip}{6pt plus 2pt minus 1pt}}
}{% if KOMA class
  \KOMAoptions{parskip=half}}
\makeatother
\usepackage{xcolor}
\IfFileExists{xurl.sty}{\usepackage{xurl}}{} % add URL line breaks if available
\IfFileExists{bookmark.sty}{\usepackage{bookmark}}{\usepackage{hyperref}}
\hypersetup{
  pdfborder={0 0 0},
  breaklinks=true}
\urlstyle{same}  % don't use monospace font for urls
\usepackage{color}
\usepackage{fancyvrb}
\newcommand{\VerbBar}{|}
\newcommand{\VERB}{\Verb[commandchars=\\\{\}]}
\DefineVerbatimEnvironment{Highlighting}{Verbatim}{commandchars=\\\{\}}
% Add ',fontsize=\small' for more characters per line
\newenvironment{Shaded}{}{}
\newcommand{\AlertTok}[1]{\textcolor[rgb]{1.00,0.00,0.00}{\textbf{#1}}}
\newcommand{\AnnotationTok}[1]{\textcolor[rgb]{0.38,0.63,0.69}{\textbf{\textit{#1}}}}
\newcommand{\AttributeTok}[1]{\textcolor[rgb]{0.49,0.56,0.16}{#1}}
\newcommand{\BaseNTok}[1]{\textcolor[rgb]{0.25,0.63,0.44}{#1}}
\newcommand{\BuiltInTok}[1]{#1}
\newcommand{\CharTok}[1]{\textcolor[rgb]{0.25,0.44,0.63}{#1}}
\newcommand{\CommentTok}[1]{\textcolor[rgb]{0.38,0.63,0.69}{\textit{#1}}}
\newcommand{\CommentVarTok}[1]{\textcolor[rgb]{0.38,0.63,0.69}{\textbf{\textit{#1}}}}
\newcommand{\ConstantTok}[1]{\textcolor[rgb]{0.53,0.00,0.00}{#1}}
\newcommand{\ControlFlowTok}[1]{\textcolor[rgb]{0.00,0.44,0.13}{\textbf{#1}}}
\newcommand{\DataTypeTok}[1]{\textcolor[rgb]{0.56,0.13,0.00}{#1}}
\newcommand{\DecValTok}[1]{\textcolor[rgb]{0.25,0.63,0.44}{#1}}
\newcommand{\DocumentationTok}[1]{\textcolor[rgb]{0.73,0.13,0.13}{\textit{#1}}}
\newcommand{\ErrorTok}[1]{\textcolor[rgb]{1.00,0.00,0.00}{\textbf{#1}}}
\newcommand{\ExtensionTok}[1]{#1}
\newcommand{\FloatTok}[1]{\textcolor[rgb]{0.25,0.63,0.44}{#1}}
\newcommand{\FunctionTok}[1]{\textcolor[rgb]{0.02,0.16,0.49}{#1}}
\newcommand{\ImportTok}[1]{#1}
\newcommand{\InformationTok}[1]{\textcolor[rgb]{0.38,0.63,0.69}{\textbf{\textit{#1}}}}
\newcommand{\KeywordTok}[1]{\textcolor[rgb]{0.00,0.44,0.13}{\textbf{#1}}}
\newcommand{\NormalTok}[1]{#1}
\newcommand{\OperatorTok}[1]{\textcolor[rgb]{0.40,0.40,0.40}{#1}}
\newcommand{\OtherTok}[1]{\textcolor[rgb]{0.00,0.44,0.13}{#1}}
\newcommand{\PreprocessorTok}[1]{\textcolor[rgb]{0.74,0.48,0.00}{#1}}
\newcommand{\RegionMarkerTok}[1]{#1}
\newcommand{\SpecialCharTok}[1]{\textcolor[rgb]{0.25,0.44,0.63}{#1}}
\newcommand{\SpecialStringTok}[1]{\textcolor[rgb]{0.73,0.40,0.53}{#1}}
\newcommand{\StringTok}[1]{\textcolor[rgb]{0.25,0.44,0.63}{#1}}
\newcommand{\VariableTok}[1]{\textcolor[rgb]{0.10,0.09,0.49}{#1}}
\newcommand{\VerbatimStringTok}[1]{\textcolor[rgb]{0.25,0.44,0.63}{#1}}
\newcommand{\WarningTok}[1]{\textcolor[rgb]{0.38,0.63,0.69}{\textbf{\textit{#1}}}}
\setlength{\emergencystretch}{3em}  % prevent overfull lines
\providecommand{\tightlist}{%
  \setlength{\itemsep}{0pt}\setlength{\parskip}{0pt}}
\setcounter{secnumdepth}{-2}
% Redefines (sub)paragraphs to behave more like sections
\ifx\paragraph\undefined\else
  \let\oldparagraph\paragraph
  \renewcommand{\paragraph}[1]{\oldparagraph{#1}\mbox{}}
\fi
\ifx\subparagraph\undefined\else
  \let\oldsubparagraph\subparagraph
  \renewcommand{\subparagraph}[1]{\oldsubparagraph{#1}\mbox{}}
\fi

% set default figure placement to htbp
\makeatletter
\def\fps@figure{htbp}
\makeatother


\date{}

\begin{document}

\hypertarget{prenom-______________________}{%
\subsubsection{Prenom :
\_\_\_\_\_\_\_\_\_\_\_\_\_\_\_\_\_\_\_\_\_\_}\label{prenom-______________________}}

\hypertarget{nom-______________________}{%
\subsubsection{Nom :
\_\_\_\_\_\_\_\_\_\_\_\_\_\_\_\_\_\_\_\_\_\_}\label{nom-______________________}}

\hypertarget{initiation-uxe0-python-questionnaire}{%
\section{Initiation à Python :
questionnaire}\label{initiation-uxe0-python-questionnaire}}

\hypertarget{question-1}{%
\subsubsection{Question 1}\label{question-1}}

Un programme contient cette instruction :

\begin{Shaded}
\begin{Highlighting}[]
\NormalTok{age }\OperatorTok{=} \BuiltInTok{input}\NormalTok{(}\StringTok{"Quel est ton âge ?"}\NormalTok{)}
\end{Highlighting}
\end{Shaded}

Pendant une execution du programme, un utilisateur répond 28 à la
question. Que contient la variable \texttt{age} ?

\begin{itemize}
\tightlist
\item
  A. \texttt{28}
\item
  B. \texttt{"28"}
\item
  C. \texttt{None}
\end{itemize}

\hypertarget{question-2}{%
\subsubsection{Question 2}\label{question-2}}

Pour concaténer un entier \texttt{n} à une chaîne de caractère, j'écris
:

\begin{itemize}
\tightlist
\item
  A. \texttt{"une\ chaîne"\ +\ str(n)}
\item
  B. \texttt{"une\ chaîne"\ +\ int(n)}
\item
  C. \texttt{"une\ chaîne"\ +\ "n"}
\end{itemize}

\hypertarget{question-3}{%
\subsubsection{Question 3}\label{question-3}}

Pour vérifier qu'une variable \texttt{n} est un entier, j'utilise :

\begin{itemize}
\tightlist
\item
  A. \texttt{isinstance(n,\ int)}
\item
  B. \texttt{n\ ==\ "int"}
\item
  C. \texttt{int(n)\ ==\ True}
\end{itemize}

\hypertarget{question-4}{%
\subsubsection{Question 4}\label{question-4}}

À la fin de l'éxecution de ce programme, que contiendra la variable
\texttt{x} ?

\begin{Shaded}
\begin{Highlighting}[]
\KeywordTok{def}\NormalTok{ dire_bonjour():}
    \BuiltInTok{print}\NormalTok{(}\StringTok{"Bonjour !"}\NormalTok{)}

\NormalTok{x }\OperatorTok{=}\NormalTok{ dire_bonjour()}
\end{Highlighting}
\end{Shaded}

\begin{itemize}
\tightlist
\item
  A. \texttt{"Bonjour\ !"}
\item
  B. \texttt{""}
\item
  C. \texttt{None}
\end{itemize}

\hypertarget{question-5}{%
\subsubsection{Question 5}\label{question-5}}

Dans le programme de la question précédente, pour que la fonction
\texttt{dire\_bonjour()} renvoie \texttt{"Bonjour\ !"}, j'aurais dû
remplacer la deuxième ligne par :

\begin{itemize}
\tightlist
\item
  A. \texttt{return\ "Bonjour\ !"}
\item
  B. \texttt{return\ print("Bonjour\ !")}
\item
  C. Rien du tout, c'était déjà bon !
\end{itemize}

\hypertarget{question-6}{%
\subsubsection{Question 6}\label{question-6}}

Pour savoir si une variable \texttt{n} contient l'entier 20, j'écris :

\begin{itemize}
\tightlist
\item
  A. \texttt{n\ =\ 20}
\item
  B. \texttt{n\ =\ "20"}
\item
  C. \texttt{n\ ==\ 20}
\item
  D. \texttt{n\ ==\ "20"}
\end{itemize}

\hypertarget{question-7} \DecValTok{2} \OperatorTok{==} \DecValTok{0}\NormalTok{:}
        \ControlFlowTok{continue}
    \BuiltInTok{print}\NormalTok{(}\StringTok{"La variable i vaut "} \OperatorTok{+} \BuiltInTok{str}\NormalTok{(i))}
\end{Highlighting}
\end{Shaded}

\begin{itemize}
\tightlist
\item
  A. Un message pour chaque entier impair entre 0 et 9 compris
\item
  B. Rien du tout
\item
  C. Il y a une erreur de syntaxe
\end{itemize}

\hypertarget{question-8} \DecValTok{2} \OperatorTok{==} \DecValTok{0}\NormalTok{:}
    \ControlFlowTok{continue}
    \BuiltInTok{print}\NormalTok{(}\StringTok{"La variable i vaut "} \OperatorTok{+} \BuiltInTok{str}\NormalTok{(i))}
\end{Highlighting}
\end{Shaded}

\begin{itemize}
\tightlist
\item
  A. Un message pour chaque entier impair entre 0 et 9 compris
\item
  B. Rien du tout
\item
  C. Il y a une erreur de syntaxe
\end{itemize}

\hypertarget{question-9}{%
\subsubsection{Question 9}\label{question-9}}

Une liste a été créée de cette manière :

\begin{Shaded}
\begin{Highlighting}[]
\NormalTok{ma_liste }\OperatorTok{=}\NormalTok{ [ }\StringTok{"Le"}\NormalTok{, }\StringTok{"Python"}\NormalTok{, }\StringTok{"c'est"}\NormalTok{, }\StringTok{"cool"}\NormalTok{, }\StringTok{"!"}\NormalTok{ ]}
\end{Highlighting}
\end{Shaded}

Laquelle de ces instructions renvoie \texttt{"Python"} ?

\begin{itemize}
\tightlist
\item
  A. \texttt{ma\_liste{[}1{]}}
\item
  B. \texttt{ma\_liste{[}2{]}}
\item
  C. \texttt{ma\_liste{[}3{]}}
\end{itemize}

\hypertarget{question-10}{%
\subsubsection{Question 10}\label{question-10}}

Pour découper une chaine de caractère \texttt{s} par rapport aux
\texttt{;} qu'elle contient et obtenir ainsi une liste, j'utilise :

\begin{itemize}
\tightlist
\item
  A. \texttt{s.strip(\textquotesingle{};\textquotesingle{})}
\item
  B. \texttt{s.join(\textquotesingle{};\textquotesingle{})}
\item
  C. \texttt{s.split(\textquotesingle{};\textquotesingle{})}
\end{itemize}

\hypertarget{question-11}{%
\subsubsection{Question 11}\label{question-11}}

Pour importer la librairie permettant de manipuler du json, j'écris au
début de mon programme :

\begin{itemize}
\tightlist
\item
  A. \texttt{include\ json}
\item
  B. \texttt{import\ json}
\item
  C. \texttt{require\ json}
\end{itemize}

\hypertarget{question-12}{%
\subsubsection{Question 12}\label{question-12}}

Ecrire une fonction \texttt{pairs} qui prend en argument une liste
d'entier et renvoie la liste entiers pairs qu'elle contient. Par
exemple, \texttt{pairs({[}3,\ 6,\ 9,\ 5,\ 2{]})} renverra
\texttt{{[}6,\ 2{]}}.

\begin{Shaded}
\begin{Highlighting}[]
\OperatorTok{|}
\OperatorTok{|}
\OperatorTok{|}
\OperatorTok{|}
\OperatorTok{|}
\OperatorTok{|}
\OperatorTok{|}
\end{Highlighting}
\end{Shaded}

\hypertarget{question-13}{%
\subsubsection{Question 13}\label{question-13}}

Des méthodes dont les noms son entourés de deux underscores
(\textbf{name}) sont:

\begin{itemize}
\tightlist
\item
  A. des méthodes spéciales qui permettent notamment d'implémenter les
  opérations de base du langage python pour ses propres classes.
\item
  B. des attributs magiques qui sont utiles pour convertir des
  propriétés en objets et objets dérivés.
\item
  C. les méthodes de la bibliothèque standard de python et téléchargées
  depuis pypi.org.
\end{itemize}

\hypertarget{question-14}{%
\subsubsection{Question 14}\label{question-14}}

Compléter ce programme pour afficher dans la console le type de
Salameche à partir du contenu de \texttt{d} :

\begin{Shaded}
\begin{Highlighting}[]
\NormalTok{d }\OperatorTok{=}\NormalTok{ [ }\StringTok{"pikachu"}\NormalTok{, }\StringTok{"salameche"}\NormalTok{ ]}

\OperatorTok{|}
\OperatorTok{|}
\OperatorTok{|}
\end{Highlighting}
\end{Shaded}

\hypertarget{question-15}{%
\subsubsection{Question 15}\label{question-15}}

Écrire la première ligne (\texttt{for\ ...}) qui permet d'itérer sur les
valeurs d'une liste \texttt{l}.

\begin{Shaded}
\begin{Highlighting}[]
\OperatorTok{|}
\OperatorTok{|}
\OperatorTok{|}
\end{Highlighting}
\end{Shaded}

\hypertarget{question-16}{%
\subsubsection{Question 16}\label{question-16}}

On veut ecrire un programme qui, étant donné une chaîne de caractère
\texttt{stuff}, essaye de la convertir en entier avec
\texttt{mon\_entier\ =\ int(stuff)}. Pour savoir si la conversion a
échoué et stocker \texttt{-1} dans \texttt{mon\_entier} à la place,
j'utilise de préférence:

\begin{itemize}
\tightlist
\item
  A. \texttt{if} et \texttt{else}
\item
  B. \texttt{isinstance}
\item
  C. \texttt{try} et \texttt{except}
\end{itemize}

\hypertarget{question-17}{%
\subsubsection{Question 17}\label{question-17}}

Ajouter, avant les deux lignes déjà écrites, quelques lignes permettant
de définir une classe \texttt{Voiture}, et que \texttt{v.color} renvoie
\texttt{"bleu"}

\begin{Shaded}
\begin{Highlighting}[]
\OperatorTok{|}
\OperatorTok{|}
\OperatorTok{|}
\OperatorTok{|}
\OperatorTok{|}
\OperatorTok{|}
\OperatorTok{|}

\NormalTok{v }\OperatorTok{=}\NormalTok{ Voiture()}
\BuiltInTok{print}\NormalTok{(v.color)}
\end{Highlighting}
\end{Shaded}

\hypertarget{question-18}{%
\subsubsection{Question 18}\label{question-18}}

En python, ce que l'on note généralement \texttt{self} représente :

\begin{itemize}
\tightlist
\item
  A. Le constructeur de la classe
\item
  B. La conversion en format JSON de l'objet
\item
  C. L'instance de objet en train d'être modifié / étudié
\end{itemize}

\hypertarget{question-19}{%
\subsubsection{Question 19}\label{question-19}}

Pour tester et rendre explicite une hypothèse faite par un programme
(par exemple tester si un nombre est positif ou si la longueur d'une
liste est 3), j'utilise :

\begin{itemize}
\tightlist
\item
  A. \texttt{if}
\item
  B. \texttt{except}
\item
  C. \texttt{assert}
\end{itemize}

\hypertarget{question-20}{%
\subsubsection{Question 20}\label{question-20}}

Pour faciliter la compréhension de mon programme par mes collègues et
mon futur moi, j'appelle mes fonctions et mes variables :

\begin{itemize}
\tightlist
\item
  A. Par des noms qui décrivent précisémment ce qu'elles font /
  contiennent
\item
  B. Avec une seule lettre comme \texttt{a}, \texttt{b}, \texttt{f},
  \texttt{x} \ldots{}
\item
  C. Obiwan Kénobi
\item
  D. La réponse D.
\end{itemize}

\end{document}
