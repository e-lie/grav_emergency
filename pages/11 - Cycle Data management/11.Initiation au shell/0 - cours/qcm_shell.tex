\PassOptionsToPackage{unicode=true}{hyperref} % options for packages loaded elsewhere
\PassOptionsToPackage{hyphens}{url}
%
\documentclass[
]{article}
\usepackage{lmodern}
\usepackage{amssymb,amsmath}
\usepackage{ifxetex,ifluatex}
\ifnum 0\ifxetex 1\fi\ifluatex 1\fi=0 % if pdftex
  \usepackage[T1]{fontenc}
  \usepackage[utf8]{inputenc}
  \usepackage{textcomp} % provides euro and other symbols
\else % if luatex or xelatex
  \usepackage{unicode-math}
  \defaultfontfeatures{Scale=MatchLowercase}
  \defaultfontfeatures[\rmfamily]{Ligatures=TeX,Scale=1}
\fi
% use upquote if available, for straight quotes in verbatim environments
\IfFileExists{upquote.sty}{\usepackage{upquote}}{}
\IfFileExists{microtype.sty}{% use microtype if available
  \usepackage[]{microtype}
  \UseMicrotypeSet[protrusion]{basicmath} % disable protrusion for tt fonts
}{}
\makeatletter
\@ifundefined{KOMAClassName}{% if non-KOMA class
  \IfFileExists{parskip.sty}{%
    \usepackage{parskip}
  }{% else
    \setlength{\parindent}{0pt}
    \setlength{\parskip}{6pt plus 2pt minus 1pt}}
}{% if KOMA class
  \KOMAoptions{parskip=half}}
\makeatother
\usepackage{xcolor}
\IfFileExists{xurl.sty}{\usepackage{xurl}}{} % add URL line breaks if available
\IfFileExists{bookmark.sty}{\usepackage{bookmark}}{\usepackage{hyperref}}
\hypersetup{
  pdfborder={0 0 0},
  breaklinks=true}
\urlstyle{same}  % don't use monospace font for urls
\setlength{\emergencystretch}{3em}  % prevent overfull lines
\providecommand{\tightlist}{%
  \setlength{\itemsep}{0pt}\setlength{\parskip}{0pt}}
\setcounter{secnumdepth}{-2}
% Redefines (sub)paragraphs to behave more like sections
\ifx\paragraph\undefined\else
  \let\oldparagraph\paragraph
  \renewcommand{\paragraph}[1]{\oldparagraph{#1}\mbox{}}
\fi
\ifx\subparagraph\undefined\else
  \let\oldsubparagraph\subparagraph
  \renewcommand{\subparagraph}[1]{\oldsubparagraph{#1}\mbox{}}
\fi

% set default figure placement to htbp
\makeatletter
\def\fps@figure{htbp}
\makeatother


\date{}

\begin{document}

\hypertarget{pruxe9nom-______________________}{%
\subsubsection{Prénom :
\_\_\_\_\_\_\_\_\_\_\_\_\_\_\_\_\_\_\_\_\_\_}\label{pruxe9nom-______________________}}

\hypertarget{nom-______________________}{%
\subsubsection{Nom :
\_\_\_\_\_\_\_\_\_\_\_\_\_\_\_\_\_\_\_\_\_\_}\label{nom-______________________}}

\hypertarget{initiation-au-shell-questionnaire}{%
\section{Initiation au shell :
questionnaire}\label{initiation-au-shell-questionnaire}}

\hypertarget{entourez-la-bonne-ruxe9ponse.}{%
\subsection{\texorpdfstring{Entourez \textbf{la} bonne
réponse.}{Entourez la bonne réponse.}}\label{entourez-la-bonne-ruxe9ponse.}}

\hypertarget{question-1}{%
\subsubsection{Question 1}\label{question-1}}

Une commande shell simple est composée \ldots{} :

\begin{itemize}
\tightlist
\item
  Du nom du programme, des éventuelles options pour préciser son
  comportement et des arguments éventuels sur lequel s'applique le
  programme.
\item
  Du nom d'une fonction suivie de parenthèses contenant les arguments
  éventuels de la commande.
\item
  Du nom du programme, d'un chemin de fichier entre parenthèses et d'un
  filtre éventuel.
\end{itemize}

\hypertarget{question-2}{%
\subsubsection{Question 2}\label{question-2}}

Un chemin absolu est:

\begin{itemize}
\tightlist
\item
  Une façon unique de désigner un fichier UNIX en décrivant comment y
  accéder depuis la racine \texttt{/}.
\item
  Un ensemble de façon de désigner un fichier UNIX en combinant
  \texttt{\textasciitilde{}}, \texttt{..} et des nom de fichiers.
\item
  Une façon unique de désigner un fichier UNIX à partir du dossier home
  \texttt{\textasciitilde{}}.
\end{itemize}

\hypertarget{question-3}{%
\subsubsection{Question 3}\label{question-3}}

Que produit la commande \texttt{chmod\ 777\ fichier.txt}

\begin{itemize}
\tightlist
\item
  \ldots{}\texttt{fichier.txt} appartiendra à \texttt{root} ne sera donc
  accessible qu'à lui.
\item
  \ldots{}\texttt{fichier.txt} deviendra un script qui pourra être
  exécuté par un shell.
\item
  \ldots{}\texttt{fichier.txt} sera accessible par tous les utilisateurs
  pour tout type d'usage.
\end{itemize}

\hypertarget{question-4}{%
\subsubsection{Question 4}\label{question-4}}

Quelles sont les trois types de permissions qui décrivent les usages
possible d'un fichier UNIX ?

\begin{itemize}
\tightlist
\item
  Permission de copier, de télécharger et de modifier.
\item
  Permission de lire, d'écrire et d'exécuter.
\item
  Permission de transfert, de filtrage et de processus.
\end{itemize}

\hypertarget{question-5}{%
\subsubsection{Question 5}\label{question-5}}

Quelle est la commande très simple qui permet de connaître depuis un
shell l'endroit où l'on se trouve dans le système de fichiers ?

\begin{center}\rule{0.5\linewidth}{\linethickness}\end{center}

\hypertarget{question-6}{%
\subsubsection{Question 6}\label{question-6}}

Quelle ligne ajoute-t-on au début d'un fichier pour en faire un script
shell ?

\begin{itemize}
\tightlist
\item
  \texttt{\#!/bin/bash}
\item
  \texttt{!!/bin/ash}
\item
  \texttt{\#/bash}
\end{itemize}

\hypertarget{question-7}{%
\subsubsection{Question 7}\label{question-7}}

Comment rediriger la sortie d'une commande vers un fichier texte ?

\begin{itemize}
\tightlist
\item
  Utiliser \texttt{\textgreater{}\ monfichier.txt} ou
  \texttt{\textgreater{}\textgreater{}\ monfichier.txt} après la
  commande.
\item
  Utiliser \texttt{wc\ -l\ monfichier.txt} ou
  \texttt{wc\ -e\ monfichier.txt} après la commande.
\item
  Utiliser \texttt{\&\&\ monfichier.txt} ou
  \texttt{\textbar{}\textbar{}\ monfichier.txt} après la commande.
\end{itemize}

\hypertarget{question-8}{%
\subsubsection{Question 8}\label{question-8}}

Pour filtrer un fichier et ne garder que les lignes contenant un certain
mot j'utilise:

\begin{itemize}
\tightlist
\item
  \texttt{find}
\item
  \texttt{grep}
\item
  \texttt{wc}
\end{itemize}

\hypertarget{question-9}{%
\subsubsection{Question 9}\label{question-9}}

Comment déclare-t-on classiquement une variable d'environnement :

\begin{itemize}
\tightlist
\item
  \texttt{\$variable\ =\ valeur;}
\item
  \texttt{export\ VAR="valeur";}
\item
  \texttt{echo\ \$VAR="valeur";}
\end{itemize}

\hypertarget{question-10}{%
\subsubsection{Question 10}\label{question-10}}

Qu'affiche \texttt{ls\ -l} ?

\begin{itemize}
\tightlist
\item
  l'ensemble des utilisateurs du système avec leux shell et leur UID.
\item
  l'ensemble des liens symbolique dans le dossier et ses sous dossiers.
\item
  l'ensemble des fichiers du dossier courant avec leurs permissions,
  possesseur et taille.
\end{itemize}

\end{document}
